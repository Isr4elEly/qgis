% Options for packages loaded elsewhere
\PassOptionsToPackage{unicode}{hyperref}
\PassOptionsToPackage{hyphens}{url}
\PassOptionsToPackage{dvipsnames,svgnames,x11names}{xcolor}
%
\documentclass[
  11pt,
  a4paper,
]{book}

\usepackage{amsmath,amssymb}
\usepackage{iftex}
\ifPDFTeX
  \usepackage[T1]{fontenc}
  \usepackage[utf8]{inputenc}
  \usepackage{textcomp} % provide euro and other symbols
\else % if luatex or xetex
  \usepackage{unicode-math}
  \defaultfontfeatures{Scale=MatchLowercase}
  \defaultfontfeatures[\rmfamily]{Ligatures=TeX,Scale=1}
\fi
\usepackage{lmodern}
\ifPDFTeX\else  
    % xetex/luatex font selection
\fi
% Use upquote if available, for straight quotes in verbatim environments
\IfFileExists{upquote.sty}{\usepackage{upquote}}{}
\IfFileExists{microtype.sty}{% use microtype if available
  \usepackage[]{microtype}
  \UseMicrotypeSet[protrusion]{basicmath} % disable protrusion for tt fonts
}{}
\makeatletter
\@ifundefined{KOMAClassName}{% if non-KOMA class
  \IfFileExists{parskip.sty}{%
    \usepackage{parskip}
  }{% else
    \setlength{\parindent}{0pt}
    \setlength{\parskip}{6pt plus 2pt minus 1pt}}
}{% if KOMA class
  \KOMAoptions{parskip=half}}
\makeatother
\usepackage{xcolor}
\usepackage[top=30mm,left=25mm,right=20mm]{geometry}
\setlength{\emergencystretch}{3em} % prevent overfull lines
\setcounter{secnumdepth}{5}
% Make \paragraph and \subparagraph free-standing
\makeatletter
\ifx\paragraph\undefined\else
  \let\oldparagraph\paragraph
  \renewcommand{\paragraph}{
    \@ifstar
      \xxxParagraphStar
      \xxxParagraphNoStar
  }
  \newcommand{\xxxParagraphStar}[1]{\oldparagraph*{#1}\mbox{}}
  \newcommand{\xxxParagraphNoStar}[1]{\oldparagraph{#1}\mbox{}}
\fi
\ifx\subparagraph\undefined\else
  \let\oldsubparagraph\subparagraph
  \renewcommand{\subparagraph}{
    \@ifstar
      \xxxSubParagraphStar
      \xxxSubParagraphNoStar
  }
  \newcommand{\xxxSubParagraphStar}[1]{\oldsubparagraph*{#1}\mbox{}}
  \newcommand{\xxxSubParagraphNoStar}[1]{\oldsubparagraph{#1}\mbox{}}
\fi
\makeatother


\providecommand{\tightlist}{%
  \setlength{\itemsep}{0pt}\setlength{\parskip}{0pt}}\usepackage{longtable,booktabs,array}
\usepackage{calc} % for calculating minipage widths
% Correct order of tables after \paragraph or \subparagraph
\usepackage{etoolbox}
\makeatletter
\patchcmd\longtable{\par}{\if@noskipsec\mbox{}\fi\par}{}{}
\makeatother
% Allow footnotes in longtable head/foot
\IfFileExists{footnotehyper.sty}{\usepackage{footnotehyper}}{\usepackage{footnote}}
\makesavenoteenv{longtable}
\usepackage{graphicx}
\makeatletter
\def\maxwidth{\ifdim\Gin@nat@width>\linewidth\linewidth\else\Gin@nat@width\fi}
\def\maxheight{\ifdim\Gin@nat@height>\textheight\textheight\else\Gin@nat@height\fi}
\makeatother
% Scale images if necessary, so that they will not overflow the page
% margins by default, and it is still possible to overwrite the defaults
% using explicit options in \includegraphics[width, height, ...]{}
\setkeys{Gin}{width=\maxwidth,height=\maxheight,keepaspectratio}
% Set default figure placement to htbp
\makeatletter
\def\fps@figure{htbp}
\makeatother

\makeatletter
\@ifpackageloaded{bookmark}{}{\usepackage{bookmark}}
\makeatother
\makeatletter
\@ifpackageloaded{caption}{}{\usepackage{caption}}
\AtBeginDocument{%
\ifdefined\contentsname
  \renewcommand*\contentsname{Índice}
\else
  \newcommand\contentsname{Índice}
\fi
\ifdefined\listfigurename
  \renewcommand*\listfigurename{Lista de Figuras}
\else
  \newcommand\listfigurename{Lista de Figuras}
\fi
\ifdefined\listtablename
  \renewcommand*\listtablename{Lista de Tabelas}
\else
  \newcommand\listtablename{Lista de Tabelas}
\fi
\ifdefined\figurename
  \renewcommand*\figurename{Figura}
\else
  \newcommand\figurename{Figura}
\fi
\ifdefined\tablename
  \renewcommand*\tablename{Tabela}
\else
  \newcommand\tablename{Tabela}
\fi
}
\@ifpackageloaded{float}{}{\usepackage{float}}
\floatstyle{ruled}
\@ifundefined{c@chapter}{\newfloat{codelisting}{h}{lop}}{\newfloat{codelisting}{h}{lop}[chapter]}
\floatname{codelisting}{Listagem}
\newcommand*\listoflistings{\listof{codelisting}{Lista de Listagens}}
\makeatother
\makeatletter
\makeatother
\makeatletter
\@ifpackageloaded{caption}{}{\usepackage{caption}}
\@ifpackageloaded{subcaption}{}{\usepackage{subcaption}}
\makeatother

\ifLuaTeX
\usepackage[bidi=basic]{babel}
\else
\usepackage[bidi=default]{babel}
\fi
\babelprovide[main,import]{brazilian}
% get rid of language-specific shorthands (see #6817):
\let\LanguageShortHands\languageshorthands
\def\languageshorthands#1{}
\ifLuaTeX
  \usepackage{selnolig}  % disable illegal ligatures
\fi
\usepackage{bookmark}

\IfFileExists{xurl.sty}{\usepackage{xurl}}{} % add URL line breaks if available
\urlstyle{same} % disable monospaced font for URLs
\hypersetup{
  pdftitle={Elaboração de mapas com o QGIS},
  pdfauthor={Israel Ely},
  pdflang={pt-br},
  colorlinks=true,
  linkcolor={Maroon},
  filecolor={Maroon},
  citecolor={Blue},
  urlcolor={Blue},
  pdfcreator={LaTeX via pandoc}}


\title{Elaboração de mapas com o QGIS}
\author{Israel Ely}
\date{quinta-feira, 20 de março de 2025}

\begin{document}
\frontmatter
\maketitle

\renewcommand*\contentsname{Índice}
{
\hypersetup{linkcolor=}
\setcounter{tocdepth}{2}
\tableofcontents
}

\mainmatter
\bookmarksetup{startatroot}

\chapter*{Preface}\label{preface}
\addcontentsline{toc}{chapter}{Preface}

\markboth{Preface}{Preface}

This is a Quarto book.

To learn more about Quarto books visit
\url{https://quarto.org/docs/books}.

\bookmarksetup{startatroot}

\chapter{APRESENTAÇÃO GERAL DO QUANTUM
GIS}\label{apresentauxe7uxe3o-geral-do-quantum-gis}

O Quantum GIS (QGIS) apresenta uma interface bastante amigável e que
pode ser completamente customizada de acordo com as suas necessidades.
Abaixo é apresentada a tela inicial do programa em sua configuração
padrão.

\part{Apresentação Geral do QGIS}

\chapter{BAIXANDO E INSTALANDO COMPLEMENTOS /
PLUGINS}\label{baixando-e-instalando-complementos-plugins}

O QGIS apresenta uma série de complementos, também chamados de plugins
que disponibilizam ao usuário uma série de funcionalidades. Vamos
começar, configurando o QGIS para baixá-los e instalá-los.

Acesse o repositório oficial dos plugins (http://pyQGIS.org/), escolha o
plugin\_installer.zip (Borys Jurgiel) e realize o download, assim como
indicado na figura abaixo.

Na versão do Quantum Gis 1.7 fornecida no DVD do treinamento este plugin
já vem instalado por padrão. A sequência abaixo mostra a forma de
instalação manual dos plugins, portanto siga para o próximo item da
apostila.

Uma vez baixado o plugin, vamos realizar a instalação. Copie o arquivo
baixado para a pasta especificada no caminho: C:\Arquivos de
programas\Quantum GIS Wroclaw\apps\QGIS\python\plugins, (todos os
plugins devem ser descompactados nesta pasta quando instalados
manualmente) e descompacte o arquivo. Agora, basta abrir o menu
Complementos / Gerenciar complementos e o plugin instalado deverá
aparecer lá, caso contrário feche e abra novamente; basta então clicar
na caixa ao lado no nome do plugin para que ele fique disponível para
utilização.

O complemento Plugin Installer permite que os complementos sejam
baixados e atualizados de forma automática.

No menu Opções acesse a aba Rede, caso sua superintendência utilize um
proxy para acessar a internet, procure saber no setor de informática
qual é o número, marque a caixa e preencha os campos com o número do
proxy, a porta, o seu nome de usuário e senha, não esqueça de escolher
no item Tipo de Proxy HttpProxy como na figura abaixo.

Em casa ou caso sua SR não utilize proxy é só desmarcar a já citada
caixa e finalizar esta etapa.

Em seguida, no menu Complementos aparecerá uma nova opção, , clique nela
e será aberta a janela abaixo que contém 3 abas: Complementos,
Repositórios, Opções; clique na aba Repositórios.

Na aba Repositórios, acione o botão ``Adicionar um grupo terceiro de
repositórios''.

Em seguida, passe para a aba Opções configure-a conforme a figura
abaixo:

Finalmente clique na aba Complementos, ela deverá mostrar vários plugins
disponíveis para instalação, cada plugin traz o nome e uma breve
descrição da sua funcionalidade, agora é só escolher e clicar no botão
instalar/atualizar complemento para que ele seja instalado.

Feito isso, basta ir novamente ao menu Complementos / Gerenciar
complementos, clicar na caixa do plugin instalado e pronto, ele deverá
estar disponível na Barra de Complementos da área de trabalho ou caso
não apareça, clique novamente no menu Complementos e ele deverá estar
habilitado.

Dando sequência, por enquanto iremos habilitar os seguintes plugins além
do Instalador de Complementos já instalado:

\chapter{CONFIGURANDO OPÇÕES}\label{configurando-opuxe7uxf5es}

Acesse a barra de menus Configurações / Opções , será aberta uma janela
com as seguintes abas:

Iniciaremos configurando a aba Geral, marque as opções conforme a figura
abaixo:

Em seguida passaremos para a aba Ferramentas de Mapa, configure as
ferramentas de medida de acordo a figura abaixo:

A próxima aba a ser configurada é Digitalizar, configure o campo Atração
para que ele funcione, como mostrado na figura abaixo:

A última aba a ser configurada é a SRC, que permite definir o sistema de
coordenadas -- projeção e datum. Importante saber que o QGIS permite a
visualização de dados em diferentes sistemas de coordenadas sem que você
tenha que transformá-los primeiro, bastando para isso ativar Habilitar
reprojeção ``on the fly'' como padrão. Até a versão anterior do QGIS,
essa transformação automática era válida apenas para dados vetoriais,
entretanto o QGIS 1.7 suporta este tipo de transformação para dados
raster.

Para configurar esta aba siga o padrão da figura abaixo:

\chapter{CONFIGURANDO AS PROPRIEDADES DO
PROJETO.}\label{configurando-as-propriedades-do-projeto.}

Para acessar as configurações do projeto acesse o menu Configurações /
``Propriedades do Projeto\ldots{}'' ou ``Crtl + Shift + P''

Será aberta uma janela com as abas abaixo:

A primeira aba das propriedades do projeto, Geral, permite dar um nome
ao projeto, definir as cores de fundo e seleção, as unidades e a
precisão do projeto (\textbf{IMPORTANTE}) neste item, escolher para o
campo Salvar caminhos: a opção Relativo pois a mesma permite que o
projeto seja salvo em HDs externos ou pendrives diminuindo problemas
quando o projeto for aberto em outras máquinas.

A segunda aba permite definir o sistema de coordenadas -- projeção e
datum. Como já explicado anteriormente ative ``Habilita transformação
SRC on the fly''.

As outras abas da configuração do projeto não necessitam ter suas
configurações padrão alteradas.

Depois dessas configurações o QGIS está pronto para ser utilizado, sendo
assim vamos Salvar o nosso projeto.

O projeto para o Quantum GIS um arquivo na extensão *.qgs que reúne as
informações sobre as camadas adicionadas, as propriedades de
visualização das camadas, a projeção e datum em que a visualização do
mapa ocorrerá e a última visualização salva das camadas. Só é possível
trabalhar com um projeto de cada vez. Acesse o menu dê o nome que
preferir e clique no botão OK.

\chapter{CRIANDO UM SRC
PERSONALIZADO.}\label{criando-um-src-personalizado.}

\part{Carregamento e visualização de dados}

\chapter{- ADICIONANDO CAMADAS
VETORIAIS}\label{adicionando-camadas-vetoriais}

O QGIS possibilita trabalhar com diversos tipos de dados vetoriais
dentre os quais podemos destacar:

**Geopakage (*.gpkg)**: GeoPackage é um formato de arquivo que permite
armazenar e compartilhar dados geoespaciais, como imagens, mapas,
atributos e geometrias. Ele foi desenvolvido pelo Open Geospatial
Consortium (OGC) e é compatível com vários dispositivos e plataformas.

Principais características: - É um formato aberto, não-proprietário e
independente de plataforma - É um contêiner SQLite 3 - Possui um esquema
de banco de dados específico - A extensão do nome do arquivo é .gpkg -
Permite armazenar vários tipos de dados geoespaciais - Facilita a
distribuição e aumenta a interoperabilidade entre plataformas

\textbf{Arquivo shape ESRI (\emph{.shp}.SHP)}: Formato nativo do
principal software comercial de Sistemas de Informação Geográfica,
geralmente é formado por pelo menos 3 arquivos com as seguintes
extensões .SHP (dados vetoriais), .DBF (banco de dados) e .SHX (arquivo
de ligação entre o . SHP e .DBF), outro arquivo que pode acompanhar
estes três e o arquivo de projeção .PRJ (nativo do principal software
comercial, mas reconhecido pelo QGIS) ou o arquivo .QPJ (nativo do QGIS)
estes dois arquivos armazenam o sistema de coordenadas e datum da
camada. Os arquivos podem ser visualizado abaixo:

\textbf{Microstation DGN (\emph{.dgn}.DGN)}: Formato do software de
Desenho Assistido por Computador (CAD) mais utilizado no INCRA;

\textbf{Valores separados por vírgula (\emph{.csv}.CSV)}: Formato
bastante leve e rápido de ser processado que pode ser produzido em
editores de texto;

\textbf{GPS eXchange Format {[}GPX{]} (\emph{.gpx}.GPX)}: Formato em que
a maioria dos programas que processam dados de GPS conseguem exportar as
informações coletados em campo;

\textbf{Keyhole Markup Language {[}KML{]} (\emph{.kml}.KML)}: Formato
produzido inicialmente para ser visualizado no software Google Earth,
diversos sites atualmente distribuem informações neste formato;

\textbf{AutoCAD DXF (\emph{.dxf}.DXF)}: Formato do principal software de
Desenho Assistido por Computador (CAD) utilizado em todo o mundo;

OBS: A forma como os arquivos vetoriais são produzidos nos programas CAD
pode dificultar a abertura dos mesmos, por exemplo, hachuras, arquivos
``atachados'', pontos e pedaços de linha que não fazem parte (``sujam'')
do mapa devem ser evitados. Quanto mais o desenho se basear em
estruturas como pontos, linhas e polígonos mais facilmente serão
reconhecidos e menor a possibilidade de conflitos.

Para visualizar tais arquivos no QGIS temos 3 opções:

I. Utilizar a Barra de Menu Camada \textgreater{} Adicionar camada
vetorial;

\chapter{ADICIONANDO CAMADAS RASTER OU
MATRICIAIS.}\label{adicionando-camadas-raster-ou-matriciais.}

O QuantumGis permite trabalhar com diversos formatos de imagem, dentre
as mais comuns estão:

\textbf{JPEG (Joint Pictures Expert Group)}: é um formato de imagem que
através de compressão elimina as informações de cores que o olho humano
não é capaz de detectar e em função disso, apesar de haver perda de
qualidade ela não é facilmente percebida, com isso os arquivos gerados
são de tamanho relativamente pequeno.

\textbf{TIFF (Tagged Image File Format)}: foi desenvolvido em 1986 em
uma tentativa de criar um padrão para imagens geradas por equipamentos
digitais. O formato é capaz de armazenar imagens em preto ou branco,
escalas de cinza e em paletas de cores com 24 ou com 32 bits.

\textbf{GeoTIFF}: é um padrão de metadados de Domínio público o qual
permite embutir informações das Coordenadas geográficas em um arquivo
TIFF. A informação adicional potencial inclui Projeções cartográficas,
Sistema de coordenadas, Elipsóides, datums, e tudo mais necessário para
estabelecer a referência espacial exata no arquivo.


\backmatter


\end{document}
